\section{Introduction}

\subsection{Project and Scientific Background}

Computing is broadly applicable, in diverse physical environments.

Recent advances in computer architecture, computer networks, and
battery technology, have enabled a proliferation of versatile,
physically mobile, Internet-enabled, computational devices. These
devices collect swathes of data about their surrounding environment,
and increasingly act upon it. Although smartphones are a prime
example, ongoing automation is putting ever more of these devices in
the air\cite{economist2017drones}, on the
roads\cite{economist2018cars}, on factory
floors\cite{spiegel2016industrie, spiegel2017industrie}, in the
fields\cite{ng2017smartfarming}, and into
workshops\cite{economist2017jets}.

Consequently, there are perpetually growing demands for rich digital
services, delivered at low latencies, to devices moving about in
diverse physical environments. These demands have thus far been met by
improving the computational capabilities of the physically mobile
devices, improving (mobile) network performance, and by having elastic
computational resources at centralized server farms (i.e., cloud
computing).

However, it remains prohibitively more expensive to move data over
physical distances, than to perform computation itself. Substantial
gains remain to be leveraged by fundamentally reducing the physical
distances that data must travel. Clearly, many digital services demand
that some data be exchanged over some physical distances.
Never-the-less, a substantial amount of time, wide-area network
traffic, and ultimately, energy, is
wasted\cite{2008-Greening-the-Internet,
2015-Distributed-Cloud-Dagstuhl,
2016-Fog-Computing-May-Help-to-Save-Energy-in-Cloud-Computing,
2017-Greening-IoT-with-Fog}, by concentrating computational resources
at physically centralized locations.

Wide-ranging physical requirements inhibit substantial expansion of
the computational capabilities of the physically mobile devices
themselves --- they must often be small in form factor, light on
weight, and powered by batteries.

% Perhaps a more readily available solution is to provision elastic
% computational resources at more diverse, yet fixed physical locations.
% For instance, our modern digital lifestyle is supported by a large
% number of heterogeneous, networked\footnote{Possibly, but not
% necessarily Internet-enabled.} devices, strategically positioned at
% fixed physical locations. For example, cell-phone towers, telco
% central
% offices\cite{2016-Central-office-re-architectured-as-a-data-center},
% WiFi-hotspots, network switches and routers.  These devices are
% subject to far less stringent physical requirements, having fixed
% power supplies and wired network connections.

% Historically, these devices have used specialized software and
% hardware. However, this approach has proven hard sustain.

% These devices have historically used specialized hardware, and
% software. However, we have grown to realize that this is expensive to
% maintain, especially in the light of dynamic network loads, and that
% there are large business opportunities in 

% this is subject to change with the advent 5G mobile
% networking, and Software-Defined Networking. Increasingly, 

% Possible locations include cell-phone towers, telco central
% offices\cite{2016-Central-office-re-architectured-as-a-data-center},
% factories, office buildings, transport hubs, libraries, retailers,
% restaurants, cafés, bars, etc.

% When a physically mobile device is in need of more computational
% resources, it can refer to some physically closest provider, rather
% than a far-away central server.

A more readily available solution is to provision elastic
computational resources at more diverse physical locations. Elements
of this solution have been considered broadly over the past decade, in
both academia and
industry\cite{2018-Survey-on-MEC-for-IoT-Realization,
2020-Edge-Computing-for-Extreme-Reliability-and-Scalability}. The
approach has bore many names, with subtle variations, including
Cloudlets\cite{2009-The-Case-for-VM-Based-Cloudlets}, Fog
Computing\cite{2012-Fog-Computing-and-Its-Role-in-IoT}, Mobile Cloud
Computing\cite{2013-MCC-A-Survey}, Mobile Edge
Computing\cite{2015-ETSI-MEC-A-key-tech-towards-5G}, and most
recently, Multi-access Edge
Computing\cite{2016-TelecomTV-MEC}\footnote{See also
\url{https://www.etsi.org/technologies/multi-access-edge-computing}.}.
However, the overarching theme has been to deliver on a physically
distributed cloud infrastructure, rather than a centralized
one\cite{2011-Distributed-Cloud, 2015-Distributed-Cloud-Dagstuhl,
2018-Ericsson-TechReview-Distributed-Cloud}, while also supporting the
physical mobility of devices leveraging such
infrastructure\cite{2018-MAGA}:

\begin{definition}[Distributed Mobile Cloud]

\label{def:mobile-cloud}

A distributed mobile cloud infrastructure, is a computational
infrastructure, where elastic computational resources, distributed
over fixed, but diverse physical locations, cater to physically mobile
devices.

\end{definition}

To fully leverage the potential benefits of a distributed mobile
cloud, we must also adapt the way we build applications, to make sure
that they can thrive in this, much more dynamic execution
environment\cite{2014-A-Survey-of-MCC-Application-Models,
2015-Distributed-Cloud-Dagstuhl,
2019-ETSI-Developing-Software-for-MEC}.

A key distinguishing feature is that it is not known at compile-time
where it would be most optimal to place each part of an application.
Optimal placement depends on a multitude of runtime conditions (e.g.,
the current location, and battery charge of mobile devices; various
device and network loads). Furthermore, these conditions change
throughout the lifetime of an application (e.g., mobile devices move),
and it may prove worthwhile to move parts of the application to new
locations, at runtime. This calls for a virtualized address space of
mobile application components, managed in a distributed fashion.

At the same time, component placement cannot be optimal without taking
the needs of the application into account. For instance, to guarantee
a certain level of service, some components must remain co-located, or
maintain some degree of physical proximity, in addition to maintaining
some computational and networking capabilities.  Deriving such
constraints is hard to fathom, without involving the programmers. For
instance, through their use of specialized programming abstractions,
whereby they provide hints and demands to the runtime, as to what
should, or should not move; when, where, and why.

Lastly, applications have something to gain from being aware of their
dynamic operating conditions. For instance, it is common for video
streaming applications to adjust their video quality to the available
network bandwidth. Furtermore, isolating resource-heavy components,
and opportunistically offloading them to other devices, is an active
area of research\cite{2013-mobile-computation-offloading}.

A programming paradigm that collectively addresses the challenges
above, may be referred to as a ``mobility-oriented'' programming
paradigm:

\begin{definition}[Mobility-Oriented Programming]

In a mobility-oriented programming paradigm, an application is
structured in terms of resource-aware, mobile components, subject to
programmer-specified mobility strategies and constraints.

\end{definition} 

The notion of mobility-oriented programming is not
novel\footnote{Although referring to this notion as
``mobility-oriented programming'' is.}. Elements of it have appeared
in the past (e.g., \cite{emerald:tocs:1988},
\cite{1998-Understanding-code-mobility}, \cite{1998-KLAIM},
\cite{2004-Context-Awareness-Mobility}). Previous work has attempted
to propose less verbose and less error-prone primitives for
programming distributed systems in general. However, mobility-oriented
programming has largely remained dormant over the past decade, as we
have witnessed an evolution of needs towards distributed mobile
clouds.

This project rests on a conjecture, that mobility-oriented
programming, adequately construed, can prove to be an \emph{eloquent}
programming paradigm for distributed mobile clouds. This conjecture is
materialized as Hypothesis \ref{hyp:main} on page \pageref{hyp:main}.
First however, a definition of eloquence:

% During that time, we have witnessed two important developments:
% First, there has been an evolution of needs towards distributed
% cloud infrastructures, as outlined above. Second, (centralized)
% cloud computing has matured, and the need to provide elastic
% computational resources, in multi-tenant environments (i.e.,
% centralized server farms), has fueled an evolution of operating
% system-level virtualization, and the emergence of so-called
% ``containers''\cite{2015-Borg}: 

% \begin{definition}[Container]

% \label{def:container}

% A container is a collection of OS-level threads, regarded by the
% operating system as a single logical unit, which may share OS-level
% resources with the rest of the system, or wield over isolated,
% virtualized instances of those resources (e.g., process table, file
% system, network interfaces).

% \end{definition}

% OS-level virtualization can be more light-weight than hardware
% virtualization. For instance, starting a container is a matter of
% spawning a collection of threads, and creating virtualized instances
% of OS-level resources, rather than starting a (virtual) machine, and
% booting an operating system from scratch.

% Perhaps more importantly, containers offer a more fine-grained
% systems structuring primitive, than do virtual machines. Unlike a
% virtual appliance, a container appliance does not need to also house
% an operating system kernel, and otherwise unrelated operating system
% resources (e.g., system files).

% These characteristics have led containers to become not only a
% popular structuring primitive for cloud-based applications, but also
% an important tool in software development. Programmers can use the
% same container appliances in development, testing, and production to
% trivially ensure consistency between these environments; while
% running fundamentally different operating systems in development,
% testing, and production. This eliminates differences in operating
% system, and software configuration as a source of discrepancy
% between these environments.

% Containerized applications communicate across container boundaries
% using interprocess communication. For instance, containers resident
% on the same machine may employ domain sockets, pipes, or message
% queues, while employing standard Internet protocols over network
% sockets for communication across (supposed) machine boundaries.

% The use of interprocess communication, allows for containerized
% applications to be polyglot, where programmers can pick the best
% programming language for each part of their application. However,
% this mode of communication can be highly error-prone, as quite
% often, no static guarantees can be provided across programming
% language boundaries.

% For all their benefits, containers are not well-braced for
% distributed clouds. While it is trivial to transfer a container
% appliance from machine to machine, live container migration remains
% experimental\footnote{See, for example, \url{https://criu.org/}.}.

\begin{definition}[Eloquence]

\label{def:eloquence}

A programming paradigm is considered eloquent for a particular domain,
if we can---with reasonable time and effort---train programmers in
using a programming technology that embodies the paradigm to quickly
deliver, largely error-free, useful, and maintainable applications
within the given domain.

\end{definition}

Of course, what may be considered ``reasonable time and effort'', or
what may be considered a ``quickly delivered'', ``useful'',
``maintainable'', or even ``largely error-free'' application, are
almost purely social questions. It follows, that it is best to apply
methods from social sciences to evaluate eloquence. How does this fit
in the context of a PhD Project in Computer
Science\footnote{Recently, a similarly inclined PhD dissertation was
submitted by Michael J. Coblenz at the Department of Computer Science
at Carnegie Mellon
University\cite{2020-PhD-User-Centerd-Design-of-Principled-PL},
advocating for a user-centered design methodology for principled
programming languages.}?

Firstly, I believe that eloquence should be put on a more rigorous
footing to advance the applicability of programming paradigm research
in practice.  Without evaluations of eloquence, programming paradigms
are rigorously distinguishable only by their conceptual differences,
and how efficient their implementations are. However, those seeking to
reap the benefits of programming paradigm research, typically have a
particular programmer resource at their disposal. It is rudimentary,
to try and make the best use of that resource.

Secondly, I believe that eloquence can be put on a more rigorous
footing, using methods already applied in other fields of Computer
Science. Empirical evaluation, including evaluation of human (e.g.,
user, programmer) factors, is today common-place in the fields of
Human-Computer
Interaction\cite{2013-HCI-An-Empirical-Research-Perspective} and
Software Engineering\cite{2007-Guide-to-Advanced-Empirical-SE}. Less
so in the field of Programming Languages, but this is subject to
ongoing developments\cite{2010-staking-claims,
2016-Programmers-Are-Users-Too,
2017-Methodological-Irregularities-in-PL-Research,
2018-Interdisciplinary-PL-Design}.

Lastly, well-founded evaluation of eloquence cannot proceed without
substantial programming efforts. Today, empirical evaluation is seldom
used to substantiate the design of programming technology. Instead,
designs tend to be driven by disjointly frantic programmer cultures
and market demands. Programming technologies, targeting similar
domains, may well differ in technical maturity, quality of
documentation, available training resources, etc. This makes them
ill-equipped for empirical evaluation, as they initially differ by
many more factors than the factor that we may choose to study.
Controlled experiments cannot be achieved without substantial masking,
re-implementation, and re-writing efforts. This is discussed at depth
in Sections \ref{sec:scientific-challenges} and
\ref{sec:scientific-method}.

\subsection{Main Research Aims and Objectives}

Resting on the above discussion, the main research aim of this project
is to:

\label{research-aim}\newcommand{\researchAim}{\begin{research-aim}
Investigate in-how-far mobility-oriented programming can be an
eloquent programming paradigm for distributed mobile
clouds.\end{research-aim}}\researchAim{}

% To that end, 4 support vectors are considered fundamental to a
% mobility-oriented programming paradigm for distributed clouds:

% \begin{enumerate}[label={(\alph*)}]

% \item \label{aim:location-transparency} That programmers can remain
% oblivious to the exact location of (parts of) data and computation at
% runtime.

% \item \label{aim:resources} That program components can maintain a
% relative capability of retrieving data, and spawning computation at
% runtime.

% \item \label{aim:move} That programmers have the capability to guide
% the runtime on when and where it might be best to move data and
% computation next.

% \item \label{aim:heterogeneity} That data and computation can be
% spread over heterogeneous devices, having different computational
% capabilities.

% \end{enumerate}

% The investigation of the relative benefits of each
% of these forms a separate research objective. 


% To get there, we can employ the following research methodology:
% Begin with an investigation of what makes programming for a
% distributed cloud ineloquent today. Such an investigation is likely
% to result in some informal hypotheses. To make those hypotheses
% concrete, consider a programming technology (or several) that
% explicitly attempts to address those issues, and one (or several)
% that does not. Prove, or disprove, the hypotheses by comparing how
% well programmers fare in solving domain-relevant problems, using the
% respective programming technologies (e.g., in a randomized
% controlled trial).

% In such a methodology, achieving the above aim, is a matter of
% accomplishing the following objectives:

Achieving the above aim is considered to be a manner of accomplishing
the following objectives:

\begin{research-objective}\label{objective:identify}

Identify key elements of eloquence in programming for a distributed
mobile cloud.

\end{research-objective}

% For instance, that programmers can remain oblivious to the exact
% location where a part of their application is executing (i.e.,
% location transparency), or that they can remain aware of the costs
% involved in reaching parts of their application (i.e., resource
% awareness).

Having identified some key elements, we can characterize one, or
several programming technologies that would provide such elements in
practice. Subsequently, objectives
\ref{objective:characterize}--\ref{objective:evaluate} may be
accomplished one, or several times:

\begin{research-objective}[Repeated] \label{objective:characterize}

Characterize a programming technology that
provides the elements of eloquence identified above.

\end{research-objective}

Implementations of the characterized programming technology may exist.
However, they may be outdated, or not suitable for subsequent
eloquence evaluations---they may be tainted by a range of objectives
other than simply addressing the particular element of eloquence in
question.

\begin{research-objective}[Repeated] \label{objective:implement}

Provide an efficient implementation of the programming technology
specified above, amenable to subsequent eloquence evaluation.

\end{research-objective}

While scientific results are achievable through mere characterization
and efficient implementation of novel programming technology alone
(roughly, objectives 2--3), to fully reach the research aim above, we
must finally also:

\begin{research-objective}[Repeated] \label{objective:evaluate}

Evaluate that the programming technology characterized and implemented
above enables programmers to eloquently leverage a distributed mobile
cloud infrastructure.

\end{research-objective}

This last research objective, is expected to result in a considerable
contribution to the research methodology in programming paradigm
research. Never-the-less, to showcase a method for scientific
evaluation of the eloquence of a programming paradigm is not the main
research aim of this project.
