\subsection{Status regarding reaching beyond the state of the art}

The onset for this project was to explore what the ideas inherent in
the Emerald programming language\cite{emerald:tocs:1988,
emerald:tse:1987, emerald:spe:1991} might be good for today, and
looking into the future. Emerald is a general purpose programming
language, devised in the early 1980s in an academic attempt to address
the challenges inherent in programming distributed systems, as they
were projected to be at the time. Emerald has since gained little
adoption, beyond small academic circles.

I have identified ``mobility-oriented programming'' as the general
programming paradigm first embodied by the Emerald programming
language. This paradigm contrasts itself with what might be called
``message-oriented programming'', such as embodied by the actor
model\cite{1973-Actors, 2016-43-Years-of-Actors}, and the Erlang
programming language\cite{2003-PhD-Armstrong}, and ``tuple-space
programming'', such as embodied by the Linda programming
language\cite{1984-Linda-and-Friends}: In mobility-oriented
programming, programmers are more concerned with the mobility of
system objects, and less, or equally, concerned about the
communication among those objects.

Mobility-oriented programming is suitable in a domain where system
objects will naturally move about, and it is important to control
their mobility to ensure some intended functioning of the application.
I have identified distributed mobile clouds as one suitable domain
of applications for such a programming paradigm. See the introduction
for a discussion of this domain.

To scientifically evaluate in-how-far such a programming paradigm is
suitable for this domain, I have formalised\footnote{In the social,
rather than mathematical sense of this word.} the notion of
programming paradigm eloquence, and designed a methodology for
evaluating it (see also Section \ref{sec:scientific-method}).

To technically enable such evaluations, I have begun work to
implement:

\begin{enumerate}

\item A modern Emerald compiler, affording experimentation with more
modern syntax for Emerald, and better error messages.

\item A modern Emerald runtime on top of BEAM, a popular Erlang
Runtime System implementation. This would allow us to run Emerald code
on-par with Erlang code.

\end{enumerate}

\subsubsection{Publications (and lack thereof)}

In order to canvas cutting edge approaches to connecting heterogeneous
components (Research Question 3), I have devised a novel
``Interconnecting Code Workshop'' at the <Programming>
conference\footnote{\url{https://programming-conference.org/}}. I have
organized two iterations of the workshop in
2019\footnote{\url{https://2019.programming-conference.org/track/icw-2019-papers}}
and
2020\footnote{\url{https://2020.programming-conference.org/home/icw-2020}},
together with my main advisor, Eric Jul. This serves as proof of the
scientific relevance of Research Question 3.

I have also had a talk approved at another workshop at the
<Programming> 2020 conference, namely MoreVMs'20 --- Workshop on
Modern Language Runtimes, Ecosystems, and VMs. The talk was entitled
``Towards Modern Runtime Support for an Object-Based Distributed
Programming Language'' and I submitted an
abstract\footnote{\url{https://2020.programming-conference.org/details/MoreVMs-2020-papers/9/Towards-Modern-Runtime-Support-for-an-Object-Based-Distributed-Programming-Language-}}.
However, I did not give the talk due to circumstances related to
COVID-19. 
