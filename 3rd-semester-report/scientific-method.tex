\section{Scientific Method}

\label{sec:scientific-method}

% The chosen scientific method seeks to address the scientific challenges outlined
% in Section \ref{sec:scientific-challenges}.

\subsection{Theoretical framework}

Having stated an overall hypothesis as to the elements of eloquence in
distributed mobile cloud programming above (Hypothesis 1), the
remainder of this project will focus on the scientific evaluation of
that hypothesis. The hypothesis may prove to be partially, or
completely false. Then, the contribution of this project would be its
research methodology, and these false results.

% This research methodology attempts to evaluate in-how-far a
% particular programming paradigm enables eloquent programming in a
% particular domain. This is done by evaluating how well programmers
% fare in that paradigm, compared to those using other paradigms, when
% given similar domain-specific tasks. This makes eloquence a social
% aspect, but conducting such social evaluations in a controlled
% manner, require substantial technical endeavours.

% For all control that can be achieved with programming technology
% design and implementation, it remains hard control for variability
% in programmer experience and motivation. Empirical evaluations of
% programmer performance 

% When evaluating programmer performance, it is hard to control for
% things like programmer motivation and previous programming experience.
% Performing randomized, controlled trials on sufficiently large
% population samples, can help mitigate for this lack of control.

\subsection{Research methodology and research methods}

Today, massively open online courses (MOOCs) have proven that they can
be used to teach, and study large populations of students\footnote{For
instance, the ACM Learning at Scale conference has been held annually
in the period 2014--2020. See also
\url{https://learningatscale.acm.org/}.}. These populations tend to
be much larger than those that can fit into conventional
classrooms\footnote{For the MOOCs studied by Jordan in
2014\cite{2014-Jordan}, of the 43,000 students that enrolled to a MOOC
on average; 6,5\% of them (2,795 students) completed the course on
average.}. At the same time, experimental studies, such as randomized,
controlled trials, are seemingly uncommon with
MOOCs\cite{2015-Rebooting-MOOC-Research}.

It is not uncommon to use web-based programming environments to
empirically evaluate programming technologies\cite{2002-web-based}.
This ensures a low barrier of entry for experiment participants,
ensures consistent programming and teaching aids, and allows to
perform randomized, controlled trials (RCTs) with less experimenter
training.

Overall, it seems like adapting the MOOC approach to conduct RCTs is a
viable approach to addressing many of the scientific challenges above.
Especially so in the wake and consequences of the COVID-19 pandemic,
which has fueled further transition to online education.

As we strive to compare popular programming paradigms with
experimental ones, we must also control for numerous unrelated
factors.  Existing programming paradigms may have efficient
implementations, good documentation, vibrant programmer communities,
and an abundance of training material. Uncontrolled comparisons of
mature and experimental programming technologies may well lead to
scientifically questionable results.

Instead, we should strive to put the mature and experimental
technologies on equal footing. This means masking and re-implementing
elements of existing programming technologies, while leveraging their
other elements to let the experimental variants achieve a similar
level of maturity. For instance, it would be viable to compare two
judiciously named variants of otherwise the same language, with code
generated for the same underlying virtual machine.

% The scientific challenges above indicated that it is hard to apply
% observational techniques to clearly identify the elements of eloquence
% in programming for distributed mobile clouds. The obvious alternative
% is to take on an experimental approach: To build hypothetical
% programming environments embodying various programming paradigms.
% Then, in a controlled fashion, teach programmers to use these
% environments, and ask them to perform similar programming tasks in the
% different programming environments. The paradigms can then be compared
% based on how well the programmers fare in the various paradigms.

% Doing this in a controlled fashion is hard.
 
% The purpose of the engineering work being to enable
% the scientific evaluation of the stated hypothesis.

% To address scientific challenges 2--4, I propose to construct a
% Massively-Open Online Course (MOOC) platform, geared to conduct the
% necessary empirical evaluations. Unlike a common MOOC platform, an in
