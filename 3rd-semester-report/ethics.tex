\section{Ethics}

This project advocates for the use of empirical evaluation involving
humans. In accordance with University of Oslo
recommendations\footnote{\url{https://www.uio.no/english/for-employees/support/research/research-data-management/data-management-plan/index.html}},
a Data Management Plan will be submitted to the Norwegian Centre for
Research Data (NSD). A draft of the Data Management Plan is attached
as Appendix \ref{app:nsd}. However, this form alone would not cover
all the necessary ethical considerations. The following covers what
remains.

Participants will engage in specially crafted programming courses,
subject to the promise that this will improve their knowledge, skills,
and competences in a particular domain. It is imperative that this
promise is delivered upon.

Furthermore, to evaluate the eloquence of a programming paradigm, we
must experiment with what turn out to be ineloquent techniques. What
is eloquent and what is not, only becomes apparent at the end of the
experiment. The solution to this problem is to have all participants
learn all the paradigms being experimented with, before the end of the
experiment, and to inform them of the results of the experiment.

For instance, if we are comparing two paradigms, A and B, and
conducting an RCT, with two groups, X and Y; we can proceed as
follows:

\begin{enumerate}

\item Let group X learn paradigm A, while group Y learns paradigm B.

\item Ask both groups to do the same task using each their paradigm.

\item Let group X now learn paradigm B, while group Y learns paradigm A.

\item Ask both groups to do the same task using each their paradigm.

\item Inform groups X and Y of the results of the experiment, so that
they know which paradigm turned out to be the most eloquent.

\end{enumerate}

This way, by the end of the experiment, no group is left unfairly
trained in what turns out to be an ineloquent programming paradigm.
