\section{Terminology}\label{app:terminology}

This document is written in accordance with the ``Guidelines for third
semester evaluation of Ph.D.-candidates at the Department of
Informatics
(IFI)''\footnote{\url{https://www.mn.uio.no/ifi/english/about/organisation/phd-committee/for-phd-candidates/3rd-semester-reporting-ifi-19.03.2018.pdf}.}.

Among other things, the guidelines ask the Ph.D.-candidate to list:

\begin{itemize}

\item Main research aims and objectives

\item Scientific challenges

\item Research questions

\end{itemize}

To state these as clearly and unambiguously as possible, this document
uses a writing style reminiscent of Mathematics and Theoretical
Computer Science---it uses typographically distinct, enumerated,
categorised blocks.

For instance, on page \pageref{research-aim}, we have:

\setcounter{research-aim}{0}

\researchAim{}

Throughout the document there are blocks like ``Research Aim 1'',
``Research Objective 2'', ``Scientific Challenge 3'', and ``Research
Question 4''. These are reminiscent of what the guidelines ask for.
However, there are also blocks that underpin them (e.g., ``Definition
1''). The purpose of this appendix is to define the blocks employed
throughout this document. In-so-doing, I also hope to clarify my
reading of the guidelines.

\bigskip

The purpose of a block is to state something as clearly, and
unambiguously as possible. It is common therefore, to have a
progression of blocks, building conceptually on one another, starting
with blocks stated in terms which can be assumed as clear and
unambiguous to the target audience. The progression employed in this
document is as follows:

\begin{description}

\item[Definition] \hfill

An explanation and demarcation of a key term to be employed in latter
text and blocks.

\item[Research Aim] \hfill

A broad statement about a desired outcome of the research project.

There may be multiple aims to a research project.

\item[Research Objective] \hfill

A broad statement about a step which, if taken, would bring us closer
to achieving a desired outcome.

We may need to accomplish multiple research objectives, to achieve a
particular desired outcome.

\item[Scientific Challenge] \hfill

An aspect of a research objective, which is important to consider in
order to achieve the objective in a scientific manner.

Scientific challenges motivate the formulation of the research
questions, and the subsequent research methodology and research
methods.

\item[Research Question] \hfill

A specific question which, if answered in a scientific manner, would
bring us closer to accomplishing a research objective.

We may need to answer multiple research questions, to accomplish a
particular research objective.

\end{description}
