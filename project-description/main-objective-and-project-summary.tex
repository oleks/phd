\section{Main Objective and Project Summary}
\label{sec:main-objective-and-project-summary}

Modern distributed systems are written in a plethora of programming
languages, run on heterogeneous hardware, and across a wide range of
runtime environments. To meet the interoperability needs of modern
distributed applications, programmers must often (manually) reduce
their communication protocols to sequencing CRUD operations (create,
read, update, delete). This programming style leaves a lot of room for
error, and leads to components that are ill-composable, and hard to
test and maintain.

Over time, programming models have emerged to tackle some of the
challenges inherent in distributed systems.  Today there exist several
specialized (1) resident libraries and frameworks, (2) dedicated
programming languages, and (3) cross-language frameworks. Resident (in
a particular language) libraries and frameworks are ill-usable across
language barriers. The dedicated programming languages have rarely
been designed with wide interoperability in mind. Perhaps, it is
because crossing language barriers is so hard, that modern
cross-language frameworks offer but a simple abstraction over CRUD ---
merely enabling cross-language remote procedure calls (RPC).

The more advanced programming models for distributed computing
\cite{miller2017dist-prog-book} remain language-locked, bound by
limited interoperability options. My objective is to advance from this
state of the art; to offer more eloquent means to orchestrate
heterogeneous components in a distributed system, through programming
language- and runtime systems engineering.
