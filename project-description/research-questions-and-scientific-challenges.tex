\section{Research Questions and Scientific Challenges}
\label{sec:research-questions-and-scientific-challenges}

In general, the research question I would like to answer is:

\begin{center}

Could a further consolidation programming language- and operating
system research help tackle the challenges of modern distributed
programming?

\end{center}

\noindent I see the following sub-questions to this question:

\begin{enumerate}

\item What is the extent of hardware and software heterogeneity in
modern distributed systems? What can we expect of the near future?

\item Is this heterogeneity unavoidable? Is it as big a problem for
modern distributed systems programming as conjectured in this
document? If not, why then is distributed programming still considered
hard?

\item From the point of view of a modern distributed systems
programmer, what are the possible benefits of a dedicated distributed
systems programming language? In particular, as opposed to a mere
standard, or a library or framework, implemented in a existing
language.

\item In-how-far do existing distributed programming languages deliver
on the possible benefits above? What could be added to these languages
to better address the issues inherent in modern distributed
programming?

\item To address heterogeneity, would it suffice to endow a existing
distributed programming language with existing options for
interoperability? For instance, provisioning support for Erlang in
Babel.

\item If not, what could be added to the existing options for
interoperability to better support the kind of interoperability needed
in modern distributed systems programming?

I see the following sub-questions to this question:

\begin{enumerate}

\item General language interoperability is hard; it is a research
direction in its own right. Is the extent of interoperation needed for
modern distributed programming, more confined than in the general
case?

\item Conversely, the challenges inherent in modern distributed
systems might not be sufficiently addressed by existing
interoperability options. For instance, with respect to reliability in
the face of hardware faults and software errors.

\item Operating-system-level virtualisation is emerging as an
essential ingredient of modern distributed systems. Could OS-level
virtualisation be leveraged to improve programming-language
interoperability? For instance, to provide security guarantees that
select programming languages, or runtime systems otherwise cannot
provide.

\end{enumerate}

\end{enumerate}

To answer these questions, I will have to analyse the architecture
existing in real-world distributed systems, implement a large amount
of software, including extending existing programming languages and
runtime systems, or implementing new ones, and conduct real-world
experiments (e.g., to showcase the possible benefits of a distributed
systems programming language).
