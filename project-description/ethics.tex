\section{Ethics}
\label{sec:ethics}

The purpose of programming languages research is to find the means for
eloquent expression of application logic, which also result in
applications that meet service-level requirements. This project, in
particular, is concerned with programming techniques that lead to
maintainable, reliable, scalable, and efficient distributed
applications.

Finding these means may involve experimenting with programming
techniques that turn out to be subpar. As discussed above, judging the
quality of a programming technique may require putting it in the hands
of unknowing programmers, and see how they fare. Especially when
the research question is concerned with a relative comparison of
different programming techniques for solving the same kinds of
problems.

In this context, it is important to thoroughly inform the test
subjects of the results, and the alternatives, so that they leave the
experiment overall as better, not worse programmers. It is important
that they do not unconsciously go on to use the techniques that we
discover to be subpar.

Involving human subjects also means that we must seek to protect their
identity. Anonymizing programmers is hard
\cite{caliskan2015anonymizing, caliskan2018when}. At the same time, it
is important to publish a sufficient amount of data to ensure
scientific integrity. Hence, it is important to seek the participants'
explicit consent, to the fact that despite our best efforts to the
contrary, elements of their programming style may leak to the public,
as part of the scientific process.
