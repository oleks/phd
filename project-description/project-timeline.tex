\section{Project Timeline}
\label{sec:project-timeline}

I choose to proceed with the 4-year program, with a 25\% stake of my
time devoted to teaching at UiO. I intend to spread this time evenly
across the 4 years, with 25\% annual working hours devoted to teaching
at UiO. The remaining 75\% of each year I intend to expend as
described below.

\bigskip

\subsection*{2018}

Erlang\cite{armstrong2003making} and
Emerald\cite{black1987distribution} are two notable general-purpose
distributed programming languages. Emerald was conceived in the early
1980s. Erlang followed half a decade later, seemingly, independently.

While the Erlang programming language today has a thriving community,
Emerald has seen little development since its inception. It is far
from clear why this rift exists, and why Erlang has emerged as the
go-to language for distributed computing. Erlang and Emerald offer
substantially different programming models, and other programming
models have also been proposed\cite{miller2017dist-prog-book}.

To make for a solid foundation for a programming language for writing
middleware, I plan to begin by devoting some time to a thorough
comparison of Erlang and Emerald, among other distributed programming
models.
 
\begin{itemize}

\item Revive the development of the Emerald programming language,
runtime, and libraries; modernise the development process.

\item Given relatively equal footing, in terms of quality of the
compiler interaction with the programmer, and prior programmer
familiarity, compare how well do Erlang and Emerald fare against one
another.

\item Study the benefits and pitfalls of the Emerald programming
model, as opposed to other distributed programming models (e.g., the
actor model).

\item Study the performance of the Emerald runtime, as opposed to the
runtimes of contemporary distributed programming languages. Consider
if Emerald could run atop a contemporary runtime (e.g., the Erlang
BEAM).

\item Study contemporary approaches to programming language
interoperability. Characterise their interface description languages.
Consider how well these approaches address efficiency, reliability,
and scalability in a distributed setting. Is RPC really insufficient?

\item Study whether operating-system-level virtualisation can be
leveraged for reliable interoperability in a distributed setting.

\item Extend Emerald's interoperability options based on the above
work.

\end{itemize}

\noindent Relevant publication venues include, but are not limited to:
Erlang Workshop and the PLMW 2018 (co-located with ICFP 2018), Curry
On 2018, Onward! 2018, USENIX ATC 2019, <Programming> 2019.

\subsection*{2019}

\begin{itemize}

\item Consider if Erlang interoperability also warrants extensions.

\item Characterise the breadth of hardware where Emerald and Erlang
can run --- either directly, or due to the interoperability extensions
added above.

\item Given relatively equal footing, in terms of quality of
implementation and documentation, an prior programmer familiarity, how
well do Erlang and Emerald, enhanced for development of heterogeneous
distributed systems, fare against one another?

\noindent Relevant publication venues include, but are not limited to:
POPL 2020, PLDI 2020.

\end{itemize}

\subsection*{2020}

\begin{itemize}

\item Change of scientific environment (1 semester).

\item Otherwise, uncharted.

\end{itemize}

\subsection*{2021}

\begin{itemize}

\item Work on dissertation.

\item Otherwise, uncharted.

\end{itemize}
